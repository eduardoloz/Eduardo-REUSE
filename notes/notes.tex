\documentclass[11pt]{article}

% Language and font encodings
\usepackage[english]{babel}
\usepackage[utf8]{inputenc}
\usepackage[T1]{fontenc}

% Sets page size and margins
\usepackage[a4paper,top=3cm,bottom=2cm,left=3cm,right=3cm,marginparwidth=1.75cm]{geometry}

% Useful packages
\usepackage{amsmath}
\usepackage{graphicx}
\usepackage[colorlinks=true, allcolors=blue]{hyperref}
\hypersetup{
     colorlinks   = true,
     citecolor    = gray
}

\usepackage{sectsty}
\usepackage{lipsum}  % Generates filler text
\usepackage[nottoc]{tocbibind} % Includes the bibliography in the table of contents


\usepackage{algorithm}
\usepackage{algpseudocode}

%for code
\usepackage{listings}

% % Bibliography setup
% \usepackage[backend=biber, style=verbose-trad2]{biblatex}
% \addbibresource{sample.bib} % Specify the bibliography file

% Custom section formatting
\sectionfont{\fontsize{12}{15}\selectfont}

% Title content
\title{CMU REUSE Notes}
\author{Eduardo Lozano}
\date{Summer 2024}

\begin{document}
\maketitle

{ \hypersetup{hidelinks} \tableofcontents }

\newpage

\section*{Introduction}
\addcontentsline{toc}{section}{Introduction}
This document contains detailed notes for the 10-week research internship focused on coding and potential paper publication.

\section{Week 1: Introduction to Project}
This
\section{Week 2: Literature Review}
% Day 1
\subsection{Day 1 - Learning decomp basics}

\textbf{Tasks:}
\begin{itemize}
  \item Make another branch on the recomp-verify repo.
  \item Make a new benchmark folder: toy\_mutex.
  \item Make a new constants folder 2-1.
  \item Create the .tla and .cfg files (see other benchmarks for examples).
  \item Decompose the files by hand.
  \item Check your work using recomp-verify.
  \item Run: \texttt{python3 ......../..../recomp-verify.py ToyMutex.tla ToyMutex.cfg --naive}.
\end{itemize}

\subsubsection*{Summary:}
Had trouble compiling because constants aren’t in config file. Java compiles faster.

\subsubsection*{Potential Concerns:}
Where to read papers / how to read effectively.

% Day 2
\subsection{Day 2 - Algorithm and Logic for State Graph Traversal}

\subsubsection*{Tasks:}
\begin{itemize}
  \item Algorithm for checking a model (use BFS).
  \item Read some articles and watch video given.
\end{itemize}

\subsubsection*{Summary:}
Article [How Amazon Web Services Uses Formal Methods] takeaways:
\begin{itemize}
  \item Formal Methods scare new methods.
  \item Formal Methods - mathematically rigorous techniques for the specification, analysis, and verification of software and hardware systems. 
  \item Safety - what the system is allowed to do.
  \item Liveness  - what the system is must eventually do.
  \item Helps with communication and all engineers understanding a description of a design.
  \item Formal Specification is not good for bugs and operator errors that cause a departure from the system’s logical intent.
  \item Surprising “sustained emergent performance degradation.”
  \item TLA+ can be used to specify an upper bound on response time, a real-world property.
  \item However, real-world infrastructure does not support hard real-time scheduling guarantees.
  \item TLA is more expressive than Alloy.
\end{itemize}
\subsection{Day 3 - Do comp-verify by hand for ToyMutex}

\subsubsection*{Tasks:}
\begin{itemize}
  \item
\end{itemize}

\subsubsection*{Summary:}
Article [How Amazon Web Services Uses Formal Methods] takeaways:
\begin{itemize}
  \item Formal Methods scare new methods.
  \item Formal Methods - mathematically rigorous techniques for the specification, analysis, and verification of software and hardware systems. 
  \item Safety - what the system is allowed to do.
  \item Liveness  - what the system is must eventually do.
  \item Helps with communication and all engineers understanding a description of a design.
  \item Formal Specification is not good for bugs and operator errors that cause a departure from the system’s logical intent.
  \item Surprising “sustained emergent performance degradation.”
  \item TLA+ can be used to specify an upper bound on response time, a real-world property.
  \item However, real-world infrastructure does not support hard real-time scheduling guarantees.
  \item TLA is more expressive than Alloy.
\end{itemize}

\begin{algorithm}
    \caption{An algorithm with caption}\label{alg:cap}
    \begin{algorithmic}
    \Require $n \geq 0$
    \Ensure $y = x^n$
    \State $y \gets 1$
    \State $X \gets x$
    \State $N \gets n$
    \While{$N \neq 0$}
    \If{$N$ is even}
        \State $X \gets X \times X$
        \State $N \gets \frac{N}{2}$  \Comment{This is a comment}
    \ElsIf{$N$ is odd}
        \State $y \gets y \times X$
        \State $N \gets N - 1$
    \EndIf
    \EndWhile
    \end{algorithmic}
\end{algorithm}

% Day 4
\subsection{Day 4 - CRA experimentation}
\subsubsection*{Tasks:}
\begin{itemize}
  \item CRA by hand with new order: C1, C2, T2. Is this more efficient?
  \item Check your work with recomp-verify, use —verbose flag to check number of states
  \item 
\end{itemize}

\subsubsection*{Summary:}
Trying to run different bits of code. What I got from doing CRA by hand and checking with the machine:
\begin{itemize}
  \item \emph{By hand:}
  \begin{itemize}
    \item From simply the $C_1 || C_2$, there ended up being no $\pi$ state meaning there could be short circuiting.
    \item Since the parallel composition operator ($||$) is commutative, $C_1||C_2||T_2 \Leftrightarrow C_1||T_2||C_2$
  \end{itemize}
  \item \emph{\emph{recom-verify checking:}}
  \begin{itemize}
    \item There were 6 states modeled with $C_1||C_2||T_2$ and 8 states modeled with 
  \end{itemize}
\end{itemize}

\subsubsection*{Summary:}
I was able to get \LaTeX working! I was able to set up the algorithm library as well
which a little too long.

% Day 5
\subsection{Day 5 - Coding}
\subsubsection*{Tasks:}
\begin{itemize}
  \item CRA by hand with new order: C1, C2, T2. Is this more efficient?
  \item Check your work with recomp-verify, use —verbose flag to check number of states
  \item 
\end{itemize}

\subsubsection*{Summary:}
Trying to run different bits of code. What I got from doing CRA by hand and checking with the machine:
\begin{itemize}
  \item \emph{By hand:}
  \begin{itemize}
    \item From simply the $C_1 || C_2$, there ended up being no $\pi$ state meaning there could be short circuiting.
    \item Since the parallel composition operator ($||$) is commutative, $C_1||C_2||T_2 \Leftrightarrow C_1||T_2||C_2$
  \end{itemize}
  \item \emph{\emph{recom-verify checking:}}
  \begin{itemize}
    \item There were 6 states modeled with $C_1||C_2||T_2$ and 8 states modeled with 
  \end{itemize}
\end{itemize}

\subsubsection*{Summary:}
I was able to get \LaTeX working! I was able to set up the algorithm library as well
which a little too long.



\section{Week 3: Initial Coding}
\subsection{Day 1 - Eduardo House Keeping}
\subsection*{Solutions to LaTeX dependencies}
\begin{lstlisting}[language=bash]
  sudo tlmgr install [package name]
\end{lstlisting}

\subsection{Day 2 - Learning decomp basics}
\subsection*{Concerns}
\begin{itemize}
  \item Shared import problems of both Will and Eduardo
  \begin{itemize}
    \item import util.ToolIO;
    \item import tlc2.tool.distributed.fp.DistributedFPSet;
    \item import tlc2.tool.distributed.TLCWorker;
    \item import org.osgi.framework.Bundle;
    \item import org.osgi.framework.BundleContext;
    \item import org.osgi.framework.FrameworkUtil;
    \item import org.osgi.service.packageadmin.PackageAdmin;
  \end{itemize}
  \item Eduardo specific issues
  \begin{itemize}
    \item import tlc2.IDistributedFPSet;\
    \item import tlc2.ITLCWorker;\ (distributed.consumer)
    \item import tlc2.IDistributedFPSet; (distributed.consumer)Z
  \end{itemize}

\end{itemize}

\subsection*{Summary}
Task for today was to try editing recomp-verify with custom.
I am using the following constants:

\begin{lstlisting}[]
  Processes == {"p1","p2","p3","p4"}
  Max == 3
  \end{lstlisting}

\begin{enumerate}
  \item \emph{``control variable''} C1;C2;T2 $\rightarrow$ 13 states and $\sim$1.74 seconds
  \item C1,C2,T2 $\rightarrow$ 20 states and 1.416/1.44 seconds
  \item if I have C1, C2; T2 $\rightarrow$ 5 states and 1.45/1.5 seconds
  \item if I have C1, T2; C2 $\rightarrow$ 46 states and 1.6 seconds
  \item if I have C1; C2; T2 $\rightarrow$ 13 states and 1.58 seconds
  \item if I have C1; T2; C2 $\rightarrow$ 20 states and 1.8 seconds

\end{enumerate}

\subsection{Day 3 - Meeting Eunsuk!}
\subsubsection*{Stand up Meeting}
\paragraph[short]{What I've worked on/working on}{
  \begin{enumerate}
    \item Learned TLA+ basics
    \item Decomposed a few examples (TwoPhaseCommit, ToyMutex)
    \item Learned how to use recomp-verify
    \item Read up on Ian's paper (understand relatively well)
  \end{enumerate}
}
\paragraph[short]{What I'll be working on}{
  \begin{enumerate}
    \item Modifying the recomp-verify code to specifically
    the number of states each component works on
    \item Actually being able to modify the code
    \item Reviewing java stuffs
  \end{enumerate}
}
\paragraph[short]{What's Blocking me}{
  \begin{enumerate}
    \item Java, Eclipse, and setting up jar files :(\
    \item Actually being able to modify the code
  \end{enumerate}
}



\section{Week 4: Advanced Coding Techniques}
\lipsum[5]

\section{Week 5: Data Analysis}

\section{Week 6: Mid-Internship Review}


\section{Week 7: Continued Research}


\section{Week 8: Preparing Results}


\section{Week 9: Discussion}


\section{Week 10: }

\section{REUSE seminars with Dr. Sunshine!}
\subsection{Week 1}
\subsection{Week 2 - Reading Research papers}
\subsubsection*{Why read papers?}
\begin{enumerate}
  \item Get a breadth of the paper.
  \begin{enumerate}
    \item See solutions and problems of the field
    \item Get an idea at high level from the Abstract and Introduction
  \end{enumerate}
  \item The work is going to be useful in pointing you in the right direction
  \begin{enumerate}
    \item Author works a lot in a particular idea $\rightarrow$ they have a good sense of prior work and identify resources
  \end{enumerate}
  \item Get ideas for reading papers
  \item Good models for writing
  \item Burrow ideas / methodologies
\end{enumerate}
\subsubsection*{How to read papers}
\begin{enumerate}
  \item Make sure to read the abstract! You can get everything you need from the abstract (most of the time)
  \begin{enumerate}
    \item Sometimes papers are wrongly cited because their title was relevant even though they had nothing to do with their citation.
  \end{enumerate}
\end{enumerate}
Know the difference between a defect, a fault, and a failure. I guess in general,
SWE terms.
$\\$
\section*{Conclusion}
\addcontentsline{toc}{section}{Concslusion}
\lipsum[13]

\newpage

% \printbibliography

\end{document}

